\documentclass[conference]{IEEEtran}

% ---------- Paquetes recomendados ----------
\usepackage[utf8]{inputenc}  % Codificación UTF-8
\usepackage[T1]{fontenc}
\usepackage{graphicx}        % Para imágenes
\usepackage{amsmath, amssymb}
\usepackage{cite}            % Para referencias IEEE
\usepackage{url}             % Para URLs

% ---------- Inicio del documento ----------
\begin{document}

% ---------- Título ----------
\title{Búsqueda adversaria para aprender a jugar a Otelo}

\author{
    \IEEEauthorblockN{Eloy Sancho Cebrero}
    \IEEEauthorblockA{
        Universidad de Sevilla \\
        Sevilla, España \\
        elosanceb@alum.us.es
    }
    \and
    \IEEEauthorblockN{Iván Fernández Limárquez}
    \IEEEauthorblockA{
        Universidad de Sevilla \\
        Sevilla, España \\
        ivaferlim@alum.us.es
    }
}

\maketitle

% ---------- Resumen ----------
\begin{abstract}
Este trabajo presenta el desarrollo de un agente inteligente capaz de jugar al juego de Otelo (Reversi) mediante la integración de técnicas de búsqueda adversaria y aprendizaje automático. Concretamente, se ha implementado el algoritmo Monte Carlo Tree Search (MCTS) con el criterio de selección Upper Confidence Bound applied to Trees (UCT), junto con una red neuronal entrenada para evaluar estados de partida. El proceso de aprendizaje se basa en la autojugabilidad del agente, permitiendo generar y etiquetar automáticamente miles de posiciones de juego. Estas posiciones se utilizan para entrenar un modelo predictivo que sustituye a la función clásica de simulación en MCTS, mejorando así la toma de decisiones del agente. Se presenta también la arquitectura del sistema, la metodología de entrenamiento y un análisis de los resultados obtenidos. El objetivo principal es explorar, en un entorno simplificado, cómo la combinación de búsqueda y redes neuronales puede producir comportamientos estratégicos eficaces sin necesidad de funciones heurísticas manuales.
\end{abstract}

% ---------- Palabras clave ----------
\begin{IEEEkeywords}
Monte Carlo Tree Search, MCTS, Upper Confidence Bound for Trees, UCT, Othello, Otelo, Reversi, Python, Inteligencia Artificial, Red neuronal
\end{IEEEkeywords}

% ---------- Secciones ----------
\section{Introducción}
En muchos juegos podemos ver IA's que nos permiten jugar contra ellas, ya sea como forma de introducirnos en el juego, o como maximo exponente del mismo.
Con el objetivo de aprender a crearlas, hemos creado una IA que aprende a jugar al juego de Otelo, para ello logramos crear un conjunto de datos con un
algoritmo base (Montecarlo) y posteriormente usamos esos datos para entrenar una red neuronal implementada en ese mismo algoritmo que pudiera perfeccionar las jugadas a lo largo de la 
partida. En este documento estudiaremos como hemos logrado implementar todo esto, además de los resultados obtenidos al enfrentar la IA con red neuronal,
contra la que no tenia. Las principales dificultades al plantearnos esta tarea fueron, la elección de la estructura de la red neuronal y sus hiperparámetros, pues para
que la red mejorará había que elegir una tasa de aprendizaje efectiva y era imperativo el hacer una elección lógica de la arquitectura, puesto que hay numerosas
formas de montar una red neuronal y cada una tiene sus ventajas e inconvenientes para cada tarea. Ademas de esto, tambien hubo dificultad en: (eloy)
\section{Preliminares}
Partimos de un agente que implementa montecarlo que se encarga de jugar partidas contra movimientos aleatorios, este guardará los estados intermedios 
en la partida y a cada uno se le asignará si, al acabar la partida, ganó, perdió o empató. Para representar estos estados hemos optado por hacer un csv, 
en el cual cada fila está formada de 65 elementos, los primeros 64 son el estado del tablero en orden, cada elemento es una casilla y si es 0 no hay nada,
si es 1 hay ficha blanca, si hay 2 hay ficha negra. El ultimo de los 65 elementos es la etiqueta que representa el resultado final de esa partida, propagado a cada estado intermedio,
-1 si perdió, 0 si empató, 1 si ganó. Tanto para enfrentarse a la IA como jugador, como para enfrentar a ambas IA entre sí, implementamos pygame, del cual se tratará su implementación mas adelante.
Para el diseño y entrenamiento de la red usamos tensorflow y keras 

\section{Implementación}
Para la implementación del proyecto se han realizado los siguientes pasos: se ha creado una versión inicial de Otelo utilizando Pygame, se ha creado un agente básico utilizando el algoritmo de Monte Carlo Tree Search, se han generado datos a partir de ese agente básico, se ha creado y, posteriormente, entrenado una red neuronal a partir de los datos generados, se ha utilizado la red neuronal como sustituta de una de las funciones de las que hace uso la implementación inicial del algoritmo MCTS y, finalmente, se ha adaptado la versión inicial de Otelo (en la que no se podía jugar contra ningún adversario) para que el agente desarrollado pudiese jugar contra el usuario.

A continuación, desarrollamos la implementación de cada uno de los pasos mencionados.

\subsection{Otelo en \texttt{pygame}}
El primer paso a seguir en la implementación del proyecto fue la creación del juego Otelo (sin presencia de un agente contra el que jugar) que nos serviría como base para crear tanto el agente MCTS como el agente final que haría uso de la red neuronal.

El juego se planteó como una matriz de ocho filas y ocho columnas en la que los jugadores realizan los cambios permitidos por las restricciones del reglamento de Otelo. En el contexto de Python, esta matriz se traduce como una lista que contiene ocho listas que, a su vez, contienen ocho números cada una y representan las ocho filas en las que se divide el tablero, siendo cada número almacenado en esa lista de listas la representación de lo que hay en cada casilla (no hay ficha, una ficha blanca o una ficha negra), haciendo un total de sesenta y cuatro números. En esta implementación de Otelo sólo hay tres posibles números que representan el color de la ficha que hay en cada casilla del tablero, el resultado final de la partida y, adicionalmente, se utilizan también para saber el turno actual de una partida en curso. Estos números son: el cero (no hay ficha en la casilla o el juego ha terminado en empate), el uno (hay una ficha blanca en una casilla, el jugador de las blancas ha ganado la partida o es el turno de dicho jugador) y el dos (hay una ficha negra en la casilla, el jugador de las blancas ha ganado la partida o es el turno de dicho jugador).

Sin embargo, esta lógica es la base que hay detrás de la interfaz gráfica mediante la cual una persona puede jugar a esta implementación de Otelo, una interfaz gráfica creada gracias a los recursos de la librería Pygame. Si bien es cierto que, entre las alternativas posibles para la implementación de este juego estaba la opción de crear un script que permitiera jugar a Otelo mediante texto (es decir, representando directamente la lógica que se ha mencionado anteriormente), valoramos más la opción de hacer uso de la librería Pygame, pues no sólo el juego sería mejor desde el punto de vista visual (aunque, dentro de los mínimos a cumplir en el proyecto, no desde el punto de vista técnico), sino que también se concibió el uso de Pygame como una manera muy intuitiva de empezar la implementación del proyecto partiendo desde cero.

Se construyó el script otelo.py para que, al ser ejecutado, inicializase una pantalla de 640 píxeles de alto y ancho en la que el color de fondo es el verde y varias líneas negras dividen las casillas cuadradas de 80 píxeles de lado. Adicionalmente, se representan las fichas como círculos blancos o negros de diámetro un poco menor al lado de las casillas y de centro situado en el centro de la casilla en la que está cada uno. Este script, como todos los videojuegos creados a partir de Pygame, contiene un bucle principal que comprueba si el usuario ha cerrado la ventana, comprueba si el usuario ha hecho clic en alguna casilla, etc. En este bucle, dentro de la implementación inicial que se trata en este apartado, sencillamente se comprueba en qué lugares de la pantalla el usuario hace clic para colocar la ficha correspondiente (en caso de que la disposición del tablero lo permita) y se alternan los turnos (\texttt{turno = 1} y \texttt{turno = 2}). Cada vez que ocurre esto, se pinta el tablero en base a la nueva matriz que lo representa y que contiene los cambios provocados por la acción del jugador (tanto la posición de la nueva ficha colocada como las fichas volteadas).

El proceso para hacer esto consiste en lo siguiente: primero es necesario convertir el píxel en el que el usuario ha hecho clic en una casilla. Para ello, se hace una división entera de cada coordenada del píxel entre la longitud en píxeles de cada casilla. Posteriormente, teniendo las coordenadas de la casilla, se comprueba si es posible colocar ficha en la casilla seleccionada. Si no lo es, no ocurre nada, pero si lo es, se coloca la ficha en la casilla y se voltean las fichas que deberán darse la vuelta tras la aparición de la nueva ficha. Para realizar tanto la comprobación de movimientos disponibles como el volteo de las fichas, se utilizan funciones que comprueban si existen fichas que serán volteadas tras la colocación de la nueva ficha (aunque, anteriormente, se comprueba si la matriz contiene un cero en el lugar seleccionado) y se guardan las coordenadas de dichas fichas para, en la modificación del tablero previa al cambio de turno, cambiar el color de las fichas a voltear; en caso de que el turno actual sea igual a 1 (blancas) se asigna el valor 1 a las celdas de la matriz equivalentes a las casillas afectadas y en caso de que el turno actual sea el del jugador de las fichas negras (igual a 2), se hace exactamente lo mismo pero asignando un 2 a las celdas correspondientes.

Por último, es necesario recalcar varios puntos: primero, antes del proceso anteriormente descrito, se comprueba si el jugador tiene movimientos disponibles. Si no los tiene, se cambia el turno. Segundo (como ya se ha mencionado) en esta implementación no existe ningún agente, el usuario coloca fichas tanto para el jugador de las blancas como para el jugador de las negras. Y tercero, es necesario recalcar que todo lo explicado es fundamental para el desarrollo del agente MCTS, pues implementan la lógica que seguirá el agente para poder jugar correctamente a Otelo.

\subsection{Motor MCTS con UCT}
Tras la implementación del juego base, se procedió a crear el que denominamos "Motor MCTS", que no es más que el algoritmo de Monte Carlo Tree Search adaptado a juego de Otelo (junto con sus funciones auxiliares) que utilizará el agente básico para poder jugar partidas y guardar los datos recogidos durante dichas partidas.

Para crear este motor de la forma más eficiente, fue necesario hacer uso de Python orientado a objetos (en otras palabras, fue necesaria la creación de una clase en Python). La clase creada es la clase \texttt{Nodo}. Esta clase es fundamental para el correcto funcionamiento del algoritmo, pues el algoritmo MCTS es simplemente un método para llevar a cabo la denominada "Búsqueda adversaria" (como lo es también el algoritmo Minimax alfa-beta) en la que creamos un árbol en el que los nodos son los estados en los que se encuentra la partida del juego en cada momento y las acciones llevadas a cabo partiendo de cada estado llevan a los distintos hijos de cada nodo, si los hay. La clase \texttt{Nodo} almacena los siguientes atributos:

\begin{itemize}
    \item \textbf{Estado}. Almacena la disposición del tablero y es lo que, en esencia, representa el nodo en cuestión.
    \item \textbf{Turno}. Almacena simplemente el turno del jugador que puede colocar ficha.
    \item \textbf{Padre}. Almacena un objeto de tipo \texttt{Nodo} que representa el nodo padre del nodo en cuestión. Por defecto, su valor es nulo.
    \item \textbf{Hijos}. Almacena una lista de objetos de tipo nodo que representa el conjunto de hijos del nodo en cuestión. A pesar de que en esta implementación ese conjunto de hijos se trata como una lista, el orden de los hijos no es realmente relevante a nivel conceptual. El valor asignado inicialmente es una lista vacía.
    \item \textbf{N}. Almacena el número de veces que el nodo ha sido visitado por las partidas simuladas. El valor asignado inicialmente es 0.
    \item \textbf{Q}. Almacena la recompensa acumulada de todas las partidas simuladas que han visitado el nodo en cuestión. El valor asignado inicialmente es 0.
    \item \textbf{Movimientos por hacer}. Almacena el conjunto de movimientos representados mediante la tupla \texttt{(fila, columna)}, disponibles para el jugador que puede colocar ficha (según el atributo \texttt{turno}). Es perfectamente posible que el jugador no tenga movimientos disponibles y que la lista esté vacía. Este atributo siempre se inicializa mediante la función \texttt{movimientos\_disponibles}, que devuelve el conjunto de jugadas válidas a partir del estado del tablero (atributo \texttt{estado}) y el turno actual (atributo \texttt{turno}).
    \item \textbf{Movimiento}. Almacena el movimiento representado mediante la tupla \texttt{(fila, columna)} que ha dado lugar al estado del nodo en cuestión. Este movimiento es nulo en el estado inicial de la partida, ya que no se ha realizado ninguna jugada previa. Por ello, su valor por defecto es \texttt{None}, y solo se especifica si el nodo representa un estado generado a partir de una jugada anterior.
\end{itemize}

Por otro lado, tenemos la función principal \texttt{mcts}, que se encarga de llevar a cabo los pasos indicados por el algoritmo MCTS. Primero crea un árbol con un nodo raíz a partir del tablero y el turno dados como parámetros. Posteriormente, realiza un número determinado de iteraciones (50 por defecto, aunque es posible pasar como parámetro otro número) y en cada iteración crea un nodo hijo a partir de la función \texttt{tree\_policy}, realiza una simulación de la partida a partir de ese nodo (mediante la función \texttt{defaut\_policy}) y retropropaga la recompensa obtenida al final de la partida simulada por el nodo y por el raíz (el nodo padre). Finalmente, tras las iteraciones especificadas, la función escoge el mejor movimiento de entre los movimientos disponibles para el nodo raíz.

La función \texttt{tree\_policy} que utiliza nuestra implementación del algoritmo MCTS se encarga de añadir a la lista de \texttt{hijos} del nodo raíz todos sus hijos (es decir, se encarga de expandir totalmente el nodo) y escoger el mejor hijo de todos, basándose en la ecuación del algoritmo Upper Confidence Bound for Trees. Para ello, recorre la lista de movimientos por hacer del nodo raíz y realiza las expansiones pertinentes utilizando la función \texttt{expand}. Esta función, dado el nodo raíz, simplemente elimina uno de los movimientos restantes de su lista de movimientos por hacer y crea el nodo que se formará a partir de la realización del movimiento especificado por la acción borrada de la lista ya mencionada (añadiendo también dicho nodo a la lista de \texttt{hijos} del nodo padre).

Una vez expandido por completo el nodo, se elige el mejor hijo mediante la función \texttt{mejor\_hijo}. Esta función, que se fundamenta en la ecuación UCT, calcula el resultado de la ecuación~\eqref{eq:uct} para cada uno de los hijos que hay en la lista de del nodo padre.

Una vez expandido por completo el nodo, se elige el mejor hijo mediante la función \texttt{mejorhijo}. Esta función, que se fundamenta en la ecuación UCT, calcula el resultado de la ecuación~\eqref{eq:uct} para cada uno de los hijos que hay en la lista de del nodo padre.

\begin{equation}
UCT = \frac{Q(v')}{N(v')} + C \cdot \sqrt{\frac{2 \cdot \ln N(v)}{N(v')}}
\label{eq:uct}
\end{equation}

\begin{itemize}
    \item $Q(v')$: recompensa acumulada del nodo hijo v'.
    \item $N(v')$: número de veces que se ha visitado el nodo hijo.
    \item $N(v)$: número de veces que se ha visitado el nodo padre v.
    \item $C$: constante de exploración que determina la importancia que le damos a explorar nodos poco visitados frente a explotar los nodos que han sido visitados bastantes veces y que son conocidos como nodos que suelen llevar a la victoria. En nuestro caso, hemos declarado el valor por defecto de $C$ en la función \texttt{mejor\_hijo} como $\sqrt{2}$ basándonos en lo indicado por los autores del artículo original de UCT. Sin embargo, es necesario tener en cuenta que el mejor valor para esta constante no tiene por qué ser este valor y que depende mucho de las circunstancias concretas del juego en cuestión.
\end{itemize}

Una vez expandido por completo el nodo raíz y escogido su mejor hijo, pasamos al funcionamiento de la función \texttt{default\_policy}. Esta función, siguiendo rigurosamente el pseudocódigo escrito en el artículo "A survey of Monte Carlo Tree Search Methods", realiza una simulación completa partiendo desde el nodo elegido como mejor hijo del raíz hasta el final de la partida, devolviendo una recompensa en función del resultado de la partida: -1 para derrotas, 0 para empates y +1 para victorias. Para realizar esta simulación, esta función no sigue construyendo el árbol realmente, se limita a elegir una acción aleatoria de la lista de acciones disponibles dado un tablero (o estado) y un turno, realiza los cambios pertinentes en el tablero y cambia de turno. Este proceso es repetido hasta que se detecta que el estado es terminal gracias a la función \texttt{no\_terminal} (función también usada en otras funciones anteriormente mencionadas). Esta función sólo comprueba que hay movimientos disponibles tanto en el turno actual como en el turno siguiente, en caso contrario, se determina que la partida ha llegado a su fin.

Tras la simulación y la obtención de la etiqueta correspondiente (-1, 0 ó 1), se retropropaga la recompensa tanto por el nodo hijo como por el nodo padre (el raíz). Para retropropagar la recompensa, realiza los siguientes cambios en los atributos \texttt{n} y \texttt{q} de cada nodo:

\begin{itemize}
    \item Para \texttt{n}: se guarda el valor \texttt{n} + 1.
    \item Para \texttt{q}: se guarda el valor \texttt{q} + $\Delta$, siendo $\Delta$ la recompensa obtenida a partir de la partida simulada por \texttt{default\_policy}
\end{itemize}

Por último, tras la realización de todas las iteraciones indicadas a la función \texttt{mcts}, se elige la mejor acción de entre todas las posibles, eligiendo el mejor hijo del nodo raíz y devolviendo el atributo \texttt{movimiento} que, como se ha mencionado anteriormente, indica el movimiento que ha dado lugar al estado del tablero que representa dicho nodo.

\subsection{Agente básico generador de datos}
Ejemplo.

Ejemplo.

\subsection{Diseño de la red neuronal}
Ejemplo.

Ejemplo.

\subsection{Entrenamiento de la red neuronal}
Ejemplo.

Ejemplo.

\subsection{Agente adversario}
Ejemplo.

Ejemplo.

\section{Pruebas y experimentación}
Explica qué datos has obtenido, cómo se comporta el agente, etc. Puedes incluir tablas, gráficos, o descripciones.

\section{Conclusiones}
Resume lo aprendido, dificultades enfrentadas, mejoras posibles.

% ---------- Bibliografía ----------
\bibliographystyle{IEEEtran}
\bibliography{bibliografia}

% ---------- Fin del documento ----------
\end{document}