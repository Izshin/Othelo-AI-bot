\documentclass[conference]{IEEEtran}

% ---------- Paquetes recomendados ----------
\usepackage[utf8]{inputenc}  % Codificación UTF-8
\usepackage[T1]{fontenc}
\usepackage{graphicx}        % Para imágenes
\usepackage{amsmath, amssymb}
\usepackage{cite}            % Para referencias IEEE
\usepackage{url}             % Para URLs

% ---------- Inicio del documento ----------
\begin{document}

% ---------- Título ----------
\title{Búsqueda adversaria para aprender a jugar a Otelo}

\author{
    \IEEEauthorblockN{Eloy Sancho Cebrero}
    \IEEEauthorblockA{
        Universidad de Sevilla \\
        Sevilla, España \\
        elosanceb@alum.us.es
    }
    \and
    \IEEEauthorblockN{Iván Fernández Limárquez}
    \IEEEauthorblockA{
        Universidad de Sevilla \\
        Sevilla, España \\
        ivaferlim@alum.us.es
    }
}

\maketitle

% ---------- Resumen ----------
\begin{abstract}
Este es el resumen del trabajo. Aquí puedes describir brevemente el objetivo, el método utilizado y los resultados más importantes. Suele ocupar unas 5–6 líneas.
\end{abstract}

% ---------- Palabras clave ----------
\begin{IEEEkeywords}
Monte Carlo Tree Search, MCTS, Upper Confidence Bound for Trees, Othello, Otelo, Reversi, Python, Inteligencia Artificial
\end{IEEEkeywords}

% ---------- Secciones ----------
\section{Introducción}
Describe aquí el contexto del proyecto, motivación y objetivos.

\section{Preliminares}
Preliminares

\section{Implementación}
Para la implementación del proyecto se han realizado los siguientes pasos: se ha creado una versión inicial de Otelo utilizando Pygame, se ha creado un agente básico utilizando el algoritmo de Monte Carlo Tree Search, se han generado datos a partir de ese agente básico, se ha creado y, posteriormente, entrenado una red neuronal a partir de los datos generados, se ha utilizado la red neuronal como sustituta de una de las funciones de las que hace uso la implementación inicial del algoritmo MCTS y, finalmente, se ha adaptado la versión inicial de Otelo (en la que no se podía jugar contra ningún adversario) para que el agente desarrollado pudiese jugar contra el usuario.

A continuación, desarrollaremos la implementación de cada uno de los pasos mencionados.

\subsection{Otelo en \texttt{pygame}}
El primer paso a seguir en la implementación del proyecto fue la creación del juego Otelo que nos serviría como base para crear tanto el agente MCTS como el agente final que haría uso de la red neuronal.



\subsection{Agente MCTS con UCT}


\subsection{Generación de datos}


\subsection{Diseño de la red neuronal}


\subsection{Entrenamiento de la red neuronal}


\subsection{Agente adversario}



\section{Pruebas y experimentación}
Explica qué datos has obtenido, cómo se comporta el agente, etc. Puedes incluir tablas, gráficos, o descripciones.

\section{Conclusiones}
Resume lo aprendido, dificultades enfrentadas, mejoras posibles.

% ---------- Bibliografía ----------
\bibliographystyle{IEEEtran}
\bibliography{bibliografia}

% ---------- Fin del documento ----------
\end{document}