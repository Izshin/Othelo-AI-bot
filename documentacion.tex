\documentclass[conference]{IEEEtran}

% ---------- Paquetes recomendados ----------
\usepackage[utf8]{inputenc}  % Codificación UTF-8
\usepackage[T1]{fontenc}
\usepackage{graphicx}        % Para imágenes
\usepackage{amsmath, amssymb}
\usepackage{cite}            % Para referencias IEEE
\usepackage{url}             % Para URLs

% ---------- Inicio del documento ----------
\begin{document}

% ---------- Título ----------
\title{Búsqueda adversaria para aprender a jugar a Otelo}

\author{
    \IEEEauthorblockN{Eloy Sancho Cebrero}
    \IEEEauthorblockA{
        Universidad de Sevilla \\
        Sevilla, España \\
        elosanceb@alum.us.es
    }
    \and
    \IEEEauthorblockN{Iván Fernández Limárquez}
    \IEEEauthorblockA{
        Universidad de Sevilla \\
        Sevilla, España \\
        ivaferlim@alum.us.es
    }
}

\maketitle

% ---------- Resumen ----------
\begin{abstract}
Este es el resumen del trabajo. Aquí puedes describir brevemente el objetivo, el método utilizado y los resultados más importantes. Suele ocupar unas 5–6 líneas.
\end{abstract}

% ---------- Palabras clave ----------
\begin{IEEEkeywords}
Monte Carlo Tree Search, MCTS, Upper Confidence Bound for Trees, Othello, Otelo, Reversi, Python, Inteligencia Artificial
\end{IEEEkeywords}

% ---------- Secciones ----------
\section{Introducción}
Describe aquí el contexto del proyecto, motivación y objetivos.

\section{Preliminares}
Preliminares

\section{Implementación}
Implementación
\begin{itemize}
    \item Implementación del Otelo en \texttt{pygame} sin agente adversario.
    \item Diseño del agente MCTS con UCT.
    \item Generación de datos para la red neuronal.
    \item Diseño e implementación de la red neuronal.
    \item Entrenamiento de la red neuronal.
    \item Implementación de agente adversario.
\end{itemize}

\section{Pruebas y experimentación}
Explica qué datos has obtenido, cómo se comporta el agente, etc. Puedes incluir tablas, gráficos, o descripciones.

\section{Conclusiones}
Resume lo aprendido, dificultades enfrentadas, mejoras posibles.

% ---------- Bibliografía ----------
\bibliographystyle{IEEEtran}
\bibliography{bibliografia}

% ---------- Fin del documento ----------
\end{document}